\documentclass{beamer}
\usepackage{graphicx,psfrag,color,upquote}
\usepackage{lmodern}

\input talk_defs.tex
\input formatting.tex

\mode<presentation>
{
\usetheme{default}
}

\newcommand{\dist}{\mathop{\bf dist{}}}

% \raggedright
% \special{! TeXDict begin /landplus90{true}store end }
% \definecolor{bluegray}{rgb}{0.15,0.20,0.40}
% \definecolor{bluegraylight}{rgb}{0.35,0.40,0.60}
% \definecolor{medgray}{rgb}{0.4,0.4,0.4}
% \definecolor{gray}{rgb}{0.35,0.35,0.35}
% \definecolor{lightgray}{rgb}{0.7,0.7,0.7}
\definecolor{darkblue}{rgb}{0.2,0.2,1.0}
\definecolor{darkgreen}{rgb}{0.0,0.5,0.3}
%\definecolor{greengray}{rgb}{0.05,0.20,0.05}
%\newcommand{\BGE}[1]{\textbf{\textcolor{bluegray}{#1}}} %bluegray emph
%\renewcommand{\labelitemi}{\textcolor{red}\textbullet}
%\renewcommand{\labelitemii}{\textcolor{red}{--}}
%\renewcommand{\end{frame} \begin{frame}}[1]{\foilhead[-1.0cm]{#1}}
%\newcommand{\end{frame} \begin{frame}}[1]{\foilhead[-1.0cm]{\textcolor{red}{#1}}}

\definecolor{pdefvalue}{rgb}{0.467,0.000,0.533}
\newcommand{\pdefvalue}[1]{\textcolor{pdefvalue}{#1}}
\definecolor{pdefidentifier}{rgb}{0.000,0.000,0.000}
\newcommand{\pdefidentifier}[1]{\textcolor{pdefidentifier}{#1}}
\definecolor{pdefcomment}{rgb}{0.502,0.502,0.502}
\newcommand{\pdefcomment}[1]{\textcolor{pdefcomment}{#1}}
\definecolor{pdefblock}{rgb}{0.267,0.533,0.867}
\newcommand{\pdefblock}[1]{\textcolor{pdefblock}{#1}}
\definecolor{pdeferror}{rgb}{0.667,0.000,0.000}
\newcommand{\pdeferror}[1]{\textcolor{pdeferror}{#1}}
\definecolor{pdefequals}{rgb}{0.000,0.490,0.000}
\newcommand{\pdefequals}[1]{\textcolor{pdefequals}{#1}}
\definecolor{pdefdim}{rgb}{0.973,0.502,0.090}
\newcommand{\pdefdim}[1]{\textcolor{pdefdim}{#1}}
\definecolor{pdefattr}{rgb}{0.000,0.000,0.467}
\newcommand{\pdefattr}[1]{\textcolor{pdefattr}{#1}}
\definecolor{pdefconstr}{rgb}{0.000,0.000,0.000}
\newcommand{\pdefconstr}[1]{\textcolor{pdefconstr}{#1}}
\definecolor{pdefobjv}{rgb}{0.000,0.000,0.000}
\newcommand{\pdefobjv}[1]{\textcolor{pdefobjv}{#1}}
\definecolor{pdefconstrsign}{rgb}{0.000,0.000,0.800}
\newcommand{\pdefconstrsign}[1]{\textcolor{pdefconstrsign}{#1}}
\definecolor{pdefrange}{rgb}{0.333,0.467,0.200}
\newcommand{\pdefrange}[1]{\textcolor{pdefrange}{#1}}
\definecolor{pdefindex}{rgb}{0.333,0.467,0.200}
\newcommand{\pdefindex}[1]{\textcolor{pdefindex}{#1}}
\definecolor{pdefbrackets}{rgb}{0.333,0.467,0.200}
\newcommand{\pdefbrackets}[1]{\textcolor{pdefbrackets}{#1}}
\definecolor{pdefsemicolon}{rgb}{0.133,0.400,0.800}
\newcommand{\pdefsemicolon}[1]{\textcolor{pdefsemicolon}{#1}}
\definecolor{pdeffunction}{rgb}{0.000,0.333,0.667}
\newcommand{\pdeffunction}[1]{\textcolor{pdeffunction}{#1}}
\definecolor{pdefdenom}{rgb}{0.000,0.333,0.667}
\newcommand{\pdefdenom}[1]{\textcolor{pdefdenom}{#1}}
\definecolor{pdefsparseindices}{rgb}{0.667,0.000,0.533}
\newcommand{\pdefsparseindices}[1]{\textcolor{pdefsparseindices}{#1}}

\title{Disciplined Convex Optimization with CVXR}


\author{\textbf{Anqi Fu} \and Bala Narasimhan \and Stephen Boyd \\[2ex]
	EE \& Statistics Departments\\[1ex]
	Stanford University}
\date{useR! Conference 2016}

\begin{document}
	
\begin{frame}
	\titlepage
\end{frame}
	
\begin{frame}{Outline}
	\tableofcontents
\end{frame}

\section{Convex Optimization}

\begin{frame}{Convex Optimization}% problem --- standard form}
	
	\[
	\begin{array}{ll} \mbox{minimize} & f_0(x)\\
	\mbox{subject to} & f_i(x) \leq 0, \quad i=1, \ldots, m\\
	& Ax=b
	\end{array}
	\]
	with variable $x \in \reals^n$
	
	\BIT
	\item Objective and inequality constraints $f_0, \ldots, f_m$ are convex %\\[1ex]
	%for all $x$, $y$, $\theta \in [0,1]$,
	%\[
	%f_i(\theta x + (1-\theta) y) \leq \theta f_i(x) + (1-\theta) f_i(y)
	%\]
	%\ie, graphs of $f_i$ curve upward
	\item Equality constraints are linear
	\EIT
	\pause
	
	\vfill
	Why?
	\BIT
	\item We can solve convex optimization problems
	\item There are many applications in many fields
	\EIT
	
\end{frame}

\iffalse
\begin{frame}{Examples}
	\BIT
	\item least-squares, least-squares with $\ell_1$ regularization (lasso)
	\item linear program (LP), quadratic program (QP)
	\item second-order cone program (SOCP)
	\item semidefinite program (SDP)
	\item maximum entropy and related problems
	\item support vector machine
	\EIT
\end{frame}
\fi

\iffalse
\begin{frame}{Convex optimization problem --- conic form}
	cone program:
	\[
	\begin{array}{ll} \mbox{minimize} & c^Tx\\
	\mbox{subject to} & Ax = b, \quad x \in \mathcal K
	\end{array}
	\]
	with variable $x \in \reals^n$
	
	\BIT
	\item linear objective, equality constraints;
	$\mathcal K$ is convex cone
	%\BIT
	%\item $x \in \mathcal K$ is a generalized nonnegativity constraint
	%\EIT
	\item special cases:
	\BIT
	\item linear program (LP)
	%second-order cone program (SOCP): $\mathcal K = $
	%$\mathcal K=\symm^n_{+}$: %(PSD matrices)
	\item semidefinite program (SDP)
	\EIT
	\vfill
	\item the modern canonical form
	\item \emph{there are well developed solvers for cone programs}
	\EIT
\end{frame}
\fi

\iffalse
\begin{frame}{Why convex optimization?}
	
	\BIT
	\item beautiful, fairly complete, and useful theory
	%\pause
	\item solution algorithms that work well in theory and practice
	\BIT
	\item convex optimization is \textbf{actionable}
	\EIT
	%\pause
	\item \textbf{many applications} in
	\BIT
	\item control
	\item combinatorial optimization
	\item signal and image processing
	\item communications, networks
	\item circuit design
	\item machine learning, statistics
	\item finance
	\EIT
	%\ldots and many more
	\EIT
\end{frame}
\fi

\iffalse
\begin{frame}{How do you solve a convex problem?}
	
	\BIT\itemsep 20pt
	\item use an existing custom solver for your specific problem
	%(\eg, SVM, lasso)
	%\item write your own (custom) solver
	%\BIT
	%\item lots of work, but can take advantage of special structure
	%\item commonly done in machine learning
	%\EIT
	\item develop a new solver for your problem using a currently
	fashionable method
	%(mirror descent, Nesterov acceleration, Frank-Wolf \ldots)
	\BIT
	\item requires work
	\item but (with luck) will scale to large problems
	\EIT
	\item transform your problem into a cone program,
	and use a standard cone program solver
	\BIT
	\item can be \emph{automated} using \emph{domain specific languages}
	\item CVX, YALMIP, CVXPY, Convex.jl
	\EIT
	
	%\BIT
	%\item extends reach of problems solvable by standard solvers
	%\item transformation can be hard to find, cumbersome to carry out
	%\EIT
	%\item \textbf{this talk:} methods to formalize and automate last
	%approach
	\EIT
\end{frame}
\fi

\section{CVXR}

\begin{frame}{CVXR}
	% Math isn't new. What's new is ability to implement in a few lines of a clear code.
	A modeling language in R for convex optimization
	
	\BIT
	\item Open source down to the solvers
	\item Uses disciplined convex programming to verify convexity
	\item Supports parameters, multiple constraints
	\item Mixes easily with general R code and other libraries
	\EIT
	
\end{frame}

\end{document}